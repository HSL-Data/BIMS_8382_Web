\documentclass[]{article}
\usepackage{lmodern}
\usepackage{amssymb,amsmath}
\usepackage{ifxetex,ifluatex}
\usepackage{fixltx2e} % provides \textsubscript
\ifnum 0\ifxetex 1\fi\ifluatex 1\fi=0 % if pdftex
  \usepackage[T1]{fontenc}
  \usepackage[utf8]{inputenc}
\else % if luatex or xelatex
  \ifxetex
    \usepackage{mathspec}
  \else
    \usepackage{fontspec}
  \fi
  \defaultfontfeatures{Ligatures=TeX,Scale=MatchLowercase}
\fi
% use upquote if available, for straight quotes in verbatim environments
\IfFileExists{upquote.sty}{\usepackage{upquote}}{}
% use microtype if available
\IfFileExists{microtype.sty}{%
\usepackage{microtype}
\UseMicrotypeSet[protrusion]{basicmath} % disable protrusion for tt fonts
}{}
\usepackage[margin=1in]{geometry}
\usepackage{hyperref}
\hypersetup{unicode=true,
            pdftitle={R: An Introduction - Homework},
            pdfborder={0 0 0},
            breaklinks=true}
\urlstyle{same}  % don't use monospace font for urls
\usepackage{color}
\usepackage{fancyvrb}
\newcommand{\VerbBar}{|}
\newcommand{\VERB}{\Verb[commandchars=\\\{\}]}
\DefineVerbatimEnvironment{Highlighting}{Verbatim}{commandchars=\\\{\}}
% Add ',fontsize=\small' for more characters per line
\usepackage{framed}
\definecolor{shadecolor}{RGB}{248,248,248}
\newenvironment{Shaded}{\begin{snugshade}}{\end{snugshade}}
\newcommand{\KeywordTok}[1]{\textcolor[rgb]{0.13,0.29,0.53}{\textbf{#1}}}
\newcommand{\DataTypeTok}[1]{\textcolor[rgb]{0.13,0.29,0.53}{#1}}
\newcommand{\DecValTok}[1]{\textcolor[rgb]{0.00,0.00,0.81}{#1}}
\newcommand{\BaseNTok}[1]{\textcolor[rgb]{0.00,0.00,0.81}{#1}}
\newcommand{\FloatTok}[1]{\textcolor[rgb]{0.00,0.00,0.81}{#1}}
\newcommand{\ConstantTok}[1]{\textcolor[rgb]{0.00,0.00,0.00}{#1}}
\newcommand{\CharTok}[1]{\textcolor[rgb]{0.31,0.60,0.02}{#1}}
\newcommand{\SpecialCharTok}[1]{\textcolor[rgb]{0.00,0.00,0.00}{#1}}
\newcommand{\StringTok}[1]{\textcolor[rgb]{0.31,0.60,0.02}{#1}}
\newcommand{\VerbatimStringTok}[1]{\textcolor[rgb]{0.31,0.60,0.02}{#1}}
\newcommand{\SpecialStringTok}[1]{\textcolor[rgb]{0.31,0.60,0.02}{#1}}
\newcommand{\ImportTok}[1]{#1}
\newcommand{\CommentTok}[1]{\textcolor[rgb]{0.56,0.35,0.01}{\textit{#1}}}
\newcommand{\DocumentationTok}[1]{\textcolor[rgb]{0.56,0.35,0.01}{\textbf{\textit{#1}}}}
\newcommand{\AnnotationTok}[1]{\textcolor[rgb]{0.56,0.35,0.01}{\textbf{\textit{#1}}}}
\newcommand{\CommentVarTok}[1]{\textcolor[rgb]{0.56,0.35,0.01}{\textbf{\textit{#1}}}}
\newcommand{\OtherTok}[1]{\textcolor[rgb]{0.56,0.35,0.01}{#1}}
\newcommand{\FunctionTok}[1]{\textcolor[rgb]{0.00,0.00,0.00}{#1}}
\newcommand{\VariableTok}[1]{\textcolor[rgb]{0.00,0.00,0.00}{#1}}
\newcommand{\ControlFlowTok}[1]{\textcolor[rgb]{0.13,0.29,0.53}{\textbf{#1}}}
\newcommand{\OperatorTok}[1]{\textcolor[rgb]{0.81,0.36,0.00}{\textbf{#1}}}
\newcommand{\BuiltInTok}[1]{#1}
\newcommand{\ExtensionTok}[1]{#1}
\newcommand{\PreprocessorTok}[1]{\textcolor[rgb]{0.56,0.35,0.01}{\textit{#1}}}
\newcommand{\AttributeTok}[1]{\textcolor[rgb]{0.77,0.63,0.00}{#1}}
\newcommand{\RegionMarkerTok}[1]{#1}
\newcommand{\InformationTok}[1]{\textcolor[rgb]{0.56,0.35,0.01}{\textbf{\textit{#1}}}}
\newcommand{\WarningTok}[1]{\textcolor[rgb]{0.56,0.35,0.01}{\textbf{\textit{#1}}}}
\newcommand{\AlertTok}[1]{\textcolor[rgb]{0.94,0.16,0.16}{#1}}
\newcommand{\ErrorTok}[1]{\textcolor[rgb]{0.64,0.00,0.00}{\textbf{#1}}}
\newcommand{\NormalTok}[1]{#1}
\usepackage{graphicx,grffile}
\makeatletter
\def\maxwidth{\ifdim\Gin@nat@width>\linewidth\linewidth\else\Gin@nat@width\fi}
\def\maxheight{\ifdim\Gin@nat@height>\textheight\textheight\else\Gin@nat@height\fi}
\makeatother
% Scale images if necessary, so that they will not overflow the page
% margins by default, and it is still possible to overwrite the defaults
% using explicit options in \includegraphics[width, height, ...]{}
\setkeys{Gin}{width=\maxwidth,height=\maxheight,keepaspectratio}
\IfFileExists{parskip.sty}{%
\usepackage{parskip}
}{% else
\setlength{\parindent}{0pt}
\setlength{\parskip}{6pt plus 2pt minus 1pt}
}
\setlength{\emergencystretch}{3em}  % prevent overfull lines
\providecommand{\tightlist}{%
  \setlength{\itemsep}{0pt}\setlength{\parskip}{0pt}}
\setcounter{secnumdepth}{0}
% Redefines (sub)paragraphs to behave more like sections
\ifx\paragraph\undefined\else
\let\oldparagraph\paragraph
\renewcommand{\paragraph}[1]{\oldparagraph{#1}\mbox{}}
\fi
\ifx\subparagraph\undefined\else
\let\oldsubparagraph\subparagraph
\renewcommand{\subparagraph}[1]{\oldsubparagraph{#1}\mbox{}}
\fi

%%% Use protect on footnotes to avoid problems with footnotes in titles
\let\rmarkdownfootnote\footnote%
\def\footnote{\protect\rmarkdownfootnote}

%%% Change title format to be more compact
\usepackage{titling}

% Create subtitle command for use in maketitle
\newcommand{\subtitle}[1]{
  \posttitle{
    \begin{center}\large#1\end{center}
    }
}

\setlength{\droptitle}{-2em}

  \title{R: An Introduction - Homework}
    \pretitle{\vspace{\droptitle}\centering\huge}
  \posttitle{\par}
    \author{}
    \preauthor{}\postauthor{}
    \date{}
    \predate{}\postdate{}
  

\begin{document}
\maketitle

\subsection{Homework 1}\label{homework-1}

Before you start, create a project folder that includes the
brauer2007\_tidy.csv dataset and your r script.

\subsubsection{EXERCISE 1}\label{exercise-1}

\subsection{R as a Calculator}\label{r-as-a-calculator}

I want to calculate how fast I drove from one location to another,
however, for some odd reason I only know how many kilometers I traveled
and the number of minutes it took me. I really only understand speed
when it comes at me in the form of miles per hour. Can you show me how
fast I traveled in miles per hour? Was I driving too fast?

\begin{enumerate}
\def\labelenumi{\arabic{enumi}.}
\tightlist
\item
  Create a variable called \texttt{kilometers} and set it equal to 120
\end{enumerate}

\begin{Shaded}
\begin{Highlighting}[]
\NormalTok{kilometers <-}\StringTok{ }\DecValTok{120}
\end{Highlighting}
\end{Shaded}

\begin{enumerate}
\def\labelenumi{\arabic{enumi}.}
\setcounter{enumi}{1}
\tightlist
\item
  Create a variable called \texttt{minutes} and set it equal to 40
\end{enumerate}

\begin{Shaded}
\begin{Highlighting}[]
\NormalTok{minutes <-}\StringTok{ }\DecValTok{40}
\end{Highlighting}
\end{Shaded}

\begin{enumerate}
\def\labelenumi{\arabic{enumi}.}
\setcounter{enumi}{2}
\tightlist
\item
  Convert \texttt{minutes} to \texttt{hours} by dividing
  \texttt{minutes} by 60
\end{enumerate}

\begin{Shaded}
\begin{Highlighting}[]
\NormalTok{hours <-}\StringTok{ }\NormalTok{minutes}\OperatorTok{/}\DecValTok{60}
\end{Highlighting}
\end{Shaded}

\begin{enumerate}
\def\labelenumi{\arabic{enumi}.}
\setcounter{enumi}{3}
\tightlist
\item
  Convert \texttt{kilometers} to \texttt{miles} by multiplying
  \texttt{kilometers} by .62
\end{enumerate}

\begin{Shaded}
\begin{Highlighting}[]
\NormalTok{miles <-}\StringTok{ }\NormalTok{kilometers }\OperatorTok{*}\StringTok{ }\NormalTok{.}\DecValTok{62}
\end{Highlighting}
\end{Shaded}

\begin{enumerate}
\def\labelenumi{\arabic{enumi}.}
\setcounter{enumi}{4}
\tightlist
\item
  Calculate the \texttt{mph} (miles per hour)
\end{enumerate}

\begin{Shaded}
\begin{Highlighting}[]
\NormalTok{mph <-}\StringTok{ }\NormalTok{miles}\OperatorTok{/}\NormalTok{hours}
\end{Highlighting}
\end{Shaded}

\section{------------------------------------------------}\label{section}

\subsubsection{EXERCISE 2}\label{exercise-2}

Calculate the cosine of the square root of the absolute value of
(2\^{}(25-21) - 40). \texttt{2}.

\begin{Shaded}
\begin{Highlighting}[]
\KeywordTok{cos}\NormalTok{(}\KeywordTok{sqrt}\NormalTok{(}\KeywordTok{abs}\NormalTok{(}\DecValTok{2}\OperatorTok{^}\NormalTok{(}\DecValTok{25} \OperatorTok{-}\StringTok{ }\DecValTok{21}\NormalTok{) }\OperatorTok{-}\StringTok{ }\DecValTok{40}\NormalTok{)))}
\end{Highlighting}
\end{Shaded}

\section{------------------------------------------------------}\label{section-1}

\subsubsection{EXERCISE 3}\label{exercise-3}

Working with vectors

\begin{enumerate}
\def\labelenumi{\arabic{enumi}.}
\tightlist
\item
  Create a vector that starts at 3 and goes to 300, increasing by 7 at
  each step.
\end{enumerate}

\begin{Shaded}
\begin{Highlighting}[]
\NormalTok{seqVec <-}\StringTok{ }\KeywordTok{seq}\NormalTok{(}\DecValTok{3}\NormalTok{, }\DecValTok{300}\NormalTok{, }\DecValTok{7}\NormalTok{)}
\end{Highlighting}
\end{Shaded}

\begin{enumerate}
\def\labelenumi{\arabic{enumi}.}
\setcounter{enumi}{1}
\tightlist
\item
  Multiply the above vector by 3 and save it as a new vector
\end{enumerate}

\begin{Shaded}
\begin{Highlighting}[]
\NormalTok{seqVec2 <-}\StringTok{ }\NormalTok{seqVec }\OperatorTok{*}\StringTok{ }\DecValTok{3}
\end{Highlighting}
\end{Shaded}

\begin{enumerate}
\def\labelenumi{\arabic{enumi}.}
\setcounter{enumi}{2}
\tightlist
\item
  Calculate the square root of the vector created in question 2. Store
  this object in a new vector.
\end{enumerate}

\begin{Shaded}
\begin{Highlighting}[]
\NormalTok{seqVec3 <-}\StringTok{ }\KeywordTok{sqrt}\NormalTok{(seqVec2)}
\end{Highlighting}
\end{Shaded}

\begin{enumerate}
\def\labelenumi{\arabic{enumi}.}
\setcounter{enumi}{3}
\tightlist
\item
  Return the first 5 elements of the vector from the object created in
  question 3.
\end{enumerate}

\begin{Shaded}
\begin{Highlighting}[]
\NormalTok{seqVec3[}\DecValTok{1}\OperatorTok{:}\DecValTok{5}\NormalTok{]}
\end{Highlighting}
\end{Shaded}

\begin{enumerate}
\def\labelenumi{\arabic{enumi}.}
\setcounter{enumi}{4}
\tightlist
\item
  Return the result of elements 3, 5, and 8 multiplied by elements 7,
  15, and 23 from the vector created in question 3.
\end{enumerate}

\begin{Shaded}
\begin{Highlighting}[]
\NormalTok{seqVec3[}\KeywordTok{c}\NormalTok{(}\DecValTok{3}\NormalTok{, }\DecValTok{5}\NormalTok{, }\DecValTok{8}\NormalTok{)] }\OperatorTok{*}\StringTok{ }\NormalTok{seqVec3[}\KeywordTok{c}\NormalTok{(}\DecValTok{7}\NormalTok{, }\DecValTok{15}\NormalTok{, }\DecValTok{23}\NormalTok{)]}
\end{Highlighting}
\end{Shaded}

\section{------------------------------------------------------}\label{section-2}

\subsubsection{EXERCISE 4}\label{exercise-4}

Working with real data

The data we're going to look at is cleaned up version of a gene
expression dataset from
\href{http://www.ncbi.nlm.nih.gov/pubmed/17959824}{Brauer et al.
Coordination of Growth Rate, Cell Cycle, Stress Response, and Metabolic
Activity in Yeast (2008) \emph{Mol Biol Cell} 19:352-367}. This data is
from a gene expression microarray, and in this paper the authors are
examining the relationship between growth rate and gene expression in
yeast cultures limited by one of six different nutrients (glucose,
leucine, ammonium, sulfate, phosphate, uracil). If you give yeast a rich
media loaded with nutrients except restrict the supply of a
\emph{single} nutrient, you can control the growth rate to any rate you
choose. By starving yeast of specific nutrients you can find genes that:

\begin{enumerate}
\def\labelenumi{\arabic{enumi}.}
\item
  \textbf{Raise or lower their expression in response to growth rate}.
  Growth-rate dependent expression patterns can tell us a lot about cell
  cycle control, and how the cell responds to stress. The authors found
  that expression of \textgreater{}25\% of all yeast genes is linearly
  correlated with growth rate, independent of the limiting nutrient.
  They also found that the subset of negatively growth-correlated genes
  is enriched for peroxisomal functions, and positively correlated genes
  mainly encode ribosomal functions.
\item
  \textbf{Respond differently when different nutrients are being
  limited}. If you see particular genes that respond very differently
  when a nutrient is sharply restricted, these genes might be involved
  in the transport or metabolism of that specific nutrient.
\item
  Open up the \texttt{brauer2007\_tidy.csv} data and inspect the data.
\item
  Import the data into R, saving it as \texttt{brauer}.
\end{enumerate}

\begin{Shaded}
\begin{Highlighting}[]
\NormalTok{ brauer <-}\StringTok{ }\KeywordTok{read.csv}\NormalTok{(}\StringTok{"Data/brauer2007_tidy.csv"}\NormalTok{, }\DataTypeTok{header =} \OtherTok{TRUE}\NormalTok{)}
\end{Highlighting}
\end{Shaded}

\begin{enumerate}
\def\labelenumi{\arabic{enumi}.}
\setcounter{enumi}{2}
\tightlist
\item
  What's the range of \texttt{expression} values represented in the
  data?
\end{enumerate}

\begin{Shaded}
\begin{Highlighting}[]
 \KeywordTok{range}\NormalTok{(brauer}\OperatorTok{$}\NormalTok{expression)}
\end{Highlighting}
\end{Shaded}

\begin{enumerate}
\def\labelenumi{\arabic{enumi}.}
\setcounter{enumi}{3}
\tightlist
\item
  What class of variable are the \texttt{bp} and \texttt{mf} columns?
\end{enumerate}

\begin{Shaded}
\begin{Highlighting}[]
   \KeywordTok{class}\NormalTok{(brauer}\OperatorTok{$}\NormalTok{bp)}
   \KeywordTok{class}\NormalTok{(brauer}\OperatorTok{$}\NormalTok{mf)}
\end{Highlighting}
\end{Shaded}

\begin{enumerate}
\def\labelenumi{\arabic{enumi}.}
\setcounter{enumi}{4}
\tightlist
\item
  What is the mean \texttt{expression} value?
\end{enumerate}

\begin{Shaded}
\begin{Highlighting}[]
   \KeywordTok{mean}\NormalTok{(brauer}\OperatorTok{$}\NormalTok{expression)}
\end{Highlighting}
\end{Shaded}

\begin{enumerate}
\def\labelenumi{\arabic{enumi}.}
\setcounter{enumi}{5}
\tightlist
\item
  How many times does Uracil show up in the \texttt{nutrient} column?
\end{enumerate}

\begin{Shaded}
\begin{Highlighting}[]
   \KeywordTok{summary}\NormalTok{(brauer}\OperatorTok{$}\NormalTok{nutrient)}
\end{Highlighting}
\end{Shaded}

\begin{enumerate}
\def\labelenumi{\arabic{enumi}.}
\setcounter{enumi}{6}
\tightlist
\item
  How many missing values are there in the \texttt{mf} column?
\end{enumerate}

\begin{Shaded}
\begin{Highlighting}[]
   \KeywordTok{summary}\NormalTok{(brauer}\OperatorTok{$}\NormalTok{mf)}
\end{Highlighting}
\end{Shaded}


\end{document}

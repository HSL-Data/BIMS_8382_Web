\documentclass[]{article}
\usepackage{lmodern}
\usepackage{amssymb,amsmath}
\usepackage{ifxetex,ifluatex}
\usepackage{fixltx2e} % provides \textsubscript
\ifnum 0\ifxetex 1\fi\ifluatex 1\fi=0 % if pdftex
  \usepackage[T1]{fontenc}
  \usepackage[utf8]{inputenc}
\else % if luatex or xelatex
  \ifxetex
    \usepackage{mathspec}
  \else
    \usepackage{fontspec}
  \fi
  \defaultfontfeatures{Ligatures=TeX,Scale=MatchLowercase}
\fi
% use upquote if available, for straight quotes in verbatim environments
\IfFileExists{upquote.sty}{\usepackage{upquote}}{}
% use microtype if available
\IfFileExists{microtype.sty}{%
\usepackage{microtype}
\UseMicrotypeSet[protrusion]{basicmath} % disable protrusion for tt fonts
}{}
\usepackage[margin=1in]{geometry}
\usepackage{hyperref}
\hypersetup{unicode=true,
            pdftitle={R: An Introduction},
            pdfborder={0 0 0},
            breaklinks=true}
\urlstyle{same}  % don't use monospace font for urls
\usepackage{color}
\usepackage{fancyvrb}
\newcommand{\VerbBar}{|}
\newcommand{\VERB}{\Verb[commandchars=\\\{\}]}
\DefineVerbatimEnvironment{Highlighting}{Verbatim}{commandchars=\\\{\}}
% Add ',fontsize=\small' for more characters per line
\usepackage{framed}
\definecolor{shadecolor}{RGB}{248,248,248}
\newenvironment{Shaded}{\begin{snugshade}}{\end{snugshade}}
\newcommand{\KeywordTok}[1]{\textcolor[rgb]{0.13,0.29,0.53}{\textbf{{#1}}}}
\newcommand{\DataTypeTok}[1]{\textcolor[rgb]{0.13,0.29,0.53}{{#1}}}
\newcommand{\DecValTok}[1]{\textcolor[rgb]{0.00,0.00,0.81}{{#1}}}
\newcommand{\BaseNTok}[1]{\textcolor[rgb]{0.00,0.00,0.81}{{#1}}}
\newcommand{\FloatTok}[1]{\textcolor[rgb]{0.00,0.00,0.81}{{#1}}}
\newcommand{\ConstantTok}[1]{\textcolor[rgb]{0.00,0.00,0.00}{{#1}}}
\newcommand{\CharTok}[1]{\textcolor[rgb]{0.31,0.60,0.02}{{#1}}}
\newcommand{\SpecialCharTok}[1]{\textcolor[rgb]{0.00,0.00,0.00}{{#1}}}
\newcommand{\StringTok}[1]{\textcolor[rgb]{0.31,0.60,0.02}{{#1}}}
\newcommand{\VerbatimStringTok}[1]{\textcolor[rgb]{0.31,0.60,0.02}{{#1}}}
\newcommand{\SpecialStringTok}[1]{\textcolor[rgb]{0.31,0.60,0.02}{{#1}}}
\newcommand{\ImportTok}[1]{{#1}}
\newcommand{\CommentTok}[1]{\textcolor[rgb]{0.56,0.35,0.01}{\textit{{#1}}}}
\newcommand{\DocumentationTok}[1]{\textcolor[rgb]{0.56,0.35,0.01}{\textbf{\textit{{#1}}}}}
\newcommand{\AnnotationTok}[1]{\textcolor[rgb]{0.56,0.35,0.01}{\textbf{\textit{{#1}}}}}
\newcommand{\CommentVarTok}[1]{\textcolor[rgb]{0.56,0.35,0.01}{\textbf{\textit{{#1}}}}}
\newcommand{\OtherTok}[1]{\textcolor[rgb]{0.56,0.35,0.01}{{#1}}}
\newcommand{\FunctionTok}[1]{\textcolor[rgb]{0.00,0.00,0.00}{{#1}}}
\newcommand{\VariableTok}[1]{\textcolor[rgb]{0.00,0.00,0.00}{{#1}}}
\newcommand{\ControlFlowTok}[1]{\textcolor[rgb]{0.13,0.29,0.53}{\textbf{{#1}}}}
\newcommand{\OperatorTok}[1]{\textcolor[rgb]{0.81,0.36,0.00}{\textbf{{#1}}}}
\newcommand{\BuiltInTok}[1]{{#1}}
\newcommand{\ExtensionTok}[1]{{#1}}
\newcommand{\PreprocessorTok}[1]{\textcolor[rgb]{0.56,0.35,0.01}{\textit{{#1}}}}
\newcommand{\AttributeTok}[1]{\textcolor[rgb]{0.77,0.63,0.00}{{#1}}}
\newcommand{\RegionMarkerTok}[1]{{#1}}
\newcommand{\InformationTok}[1]{\textcolor[rgb]{0.56,0.35,0.01}{\textbf{\textit{{#1}}}}}
\newcommand{\WarningTok}[1]{\textcolor[rgb]{0.56,0.35,0.01}{\textbf{\textit{{#1}}}}}
\newcommand{\AlertTok}[1]{\textcolor[rgb]{0.94,0.16,0.16}{{#1}}}
\newcommand{\ErrorTok}[1]{\textcolor[rgb]{0.64,0.00,0.00}{\textbf{{#1}}}}
\newcommand{\NormalTok}[1]{{#1}}
\usepackage{graphicx,grffile}
\makeatletter
\def\maxwidth{\ifdim\Gin@nat@width>\linewidth\linewidth\else\Gin@nat@width\fi}
\def\maxheight{\ifdim\Gin@nat@height>\textheight\textheight\else\Gin@nat@height\fi}
\makeatother
% Scale images if necessary, so that they will not overflow the page
% margins by default, and it is still possible to overwrite the defaults
% using explicit options in \includegraphics[width, height, ...]{}
\setkeys{Gin}{width=\maxwidth,height=\maxheight,keepaspectratio}
\IfFileExists{parskip.sty}{%
\usepackage{parskip}
}{% else
\setlength{\parindent}{0pt}
\setlength{\parskip}{6pt plus 2pt minus 1pt}
}
\setlength{\emergencystretch}{3em}  % prevent overfull lines
\providecommand{\tightlist}{%
  \setlength{\itemsep}{0pt}\setlength{\parskip}{0pt}}
\setcounter{secnumdepth}{0}
% Redefines (sub)paragraphs to behave more like sections
\ifx\paragraph\undefined\else
\let\oldparagraph\paragraph
\renewcommand{\paragraph}[1]{\oldparagraph{#1}\mbox{}}
\fi
\ifx\subparagraph\undefined\else
\let\oldsubparagraph\subparagraph
\renewcommand{\subparagraph}[1]{\oldsubparagraph{#1}\mbox{}}
\fi

%%% Use protect on footnotes to avoid problems with footnotes in titles
\let\rmarkdownfootnote\footnote%
\def\footnote{\protect\rmarkdownfootnote}

%%% Change title format to be more compact
\usepackage{titling}

% Create subtitle command for use in maketitle
\newcommand{\subtitle}[1]{
  \posttitle{
    \begin{center}\large#1\end{center}
    }
}

\setlength{\droptitle}{-2em}

  \title{R: An Introduction}
    \pretitle{\vspace{\droptitle}\centering\huge}
  \posttitle{\par}
    \author{}
    \preauthor{}\postauthor{}
    \date{}
    \predate{}\postdate{}
  

\begin{document}
\maketitle

\href{Week_1_RMarkdown.html}{Part 2 of Week 1, using RMarkdown}.

\subsection{Introduction to RStudio}\label{introduction-to-rstudio}

This section introduces the R environment and some of the most basic
funcionality aspects of R that are used through the remainder of the
class. This section assumes little to no experience with statistical
computing with R. We will introduce the R statistical computing
environment, RStudio, and the dataset that we will work with for the
remainder of the lesson. We will cover very basic functionality in R,
including variables, functions, and importing/inspecting data frames.

R is the underlying statistical computing environment RStudio sits on
top of R and makes writing code a lot easier.

\subsubsection{R Studio Panes}\label{r-studio-panes}

On the top left is the script or editor window. This is where we are
going to write all of our code. On the lower left there's the console
window. This is where R tells us what it thinks we told it and then the
answer. This is basically the same kind of interface as the terminal The
top right has the environment and history tabs \ldots{} environment is a
list of all objects that are saved in memory \ldots{} history is the
history of all commands that have been run \ldots{} On the bottom right
hand side there's a window with Files / Plots / Help

\subsection{Creating projects}\label{creating-projects}

Projects provide a way for the user to organize their data, code, and
any other documents in one place. To create a project you go to File
-\textgreater{} New Project and then create a new project which will
create a new folder for your data/script, etc. or you can choose an
existing project which will allow you to point to a folder that already
has all of your resources.

\subsection{Comments}\label{comments}

Comments are an essential part in understanding what your code does and
why something was coded a specific way. Comments are notes to yourself
or to others that are made to explain what is going on in the code. They
are not read by R and will only be seen when looking at the script.

\begin{Shaded}
\begin{Highlighting}[]
\CommentTok{#Intro to R}
\CommentTok{#------------------------------------------------------}
\CommentTok{#this is a comment and will not be run as code}
\end{Highlighting}
\end{Shaded}

\subsection{Basics of R}\label{basics-of-r}

R can be used to do very advanced calculations, statistics, and data
wrangling, however, it can also be used to do some very basic
calculations and functions. R can act like a typical calculator, but it
can also store the values that are run for later. To run lines of code
in R you can put the cursor on the line of code you wish to run and hit
the ``Run'' button. Alternatively you can highlight the line or lines of
code you want to run and hit the ``Run'' button. You can also use the
keyboard shortcut of \texttt{Ctrl}+\texttt{Enter} on Windows or
\texttt{CMD}+\texttt{Enter} on a MAC.

\begin{Shaded}
\begin{Highlighting}[]
\CommentTok{# R can be used as a calculator}
\DecValTok{2} \NormalTok{+}\StringTok{ }\DecValTok{2}
\DecValTok{5} \NormalTok{*}\StringTok{ }\DecValTok{3}
\DecValTok{2} \NormalTok{^}\StringTok{ }\DecValTok{5}

\CommentTok{# R knows order of operations}
\DecValTok{2} \NormalTok{+}\StringTok{ }\DecValTok{3}\NormalTok{/}\DecValTok{4} \NormalTok{*}\StringTok{ }\DecValTok{4} \NormalTok{*}\StringTok{ }\NormalTok{(}\DecValTok{4+4}\NormalTok{)/}\DecValTok{4}\NormalTok{^}\DecValTok{2}
\end{Highlighting}
\end{Shaded}

As mentioned before, to do useful and interesting things, we need to
assign \emph{values} to \emph{objects}. To create objects, we need to
give it a name followed by the assignment operator \texttt{\textless{}-}
and the value we want to give it:

\begin{Shaded}
\begin{Highlighting}[]
\CommentTok{# Assign objects using <-}
\NormalTok{weight_kg <-}\StringTok{ }\DecValTok{55}
\end{Highlighting}
\end{Shaded}

\texttt{\textless{}-} is the assignment operator. Assigns values on the
right to objects on the left, it is like an arrow that points from the
value to the object. Mostly similar to \texttt{=} but not always. Learn
to use \texttt{\textless{}-} as it is good programming practice. Using
\texttt{=} in place of \texttt{\textless{}-} can lead to issues down the
line. The keyboard shortcut for inserting the \texttt{\textless{}-}
operator is \texttt{Alt-dash}.

Objects can be given any name such as \texttt{x},
\texttt{current\_temperature}, or \texttt{subject\_id}. You want your
object names to be explicit and not too long. They cannot start with a
number (\texttt{2x} is not valid but \texttt{x2} is). R is case
sensitive (e.g., \texttt{weight\_kg} is different from
\texttt{Weight\_kg}). There are some names that cannot be used because
they represent the names of fundamental functions in R (e.g.,
\texttt{if}, \texttt{else}, \texttt{for}, see
\href{https://stat.ethz.ch/R-manual/R-devel/library/base/html/Reserved.html}{here}
for a complete list). In general, even if it's allowed, it's best to not
use other function names, which we'll get into shortly (e.g.,
\texttt{c}, \texttt{T}, \texttt{mean}, \texttt{data}, \texttt{df},
\texttt{weights}). In doubt check the help to see if the name is already
in use. It's also best to avoid dots (\texttt{.}) within a variable name
as in \texttt{my.dataset}. It is also recommended to use nouns for
variable names, and verbs for function names.

When assigning a value to an object, R does not print anything. You can
force to print the value by typing the name:

\begin{Shaded}
\begin{Highlighting}[]
\NormalTok{weight_kg}

\CommentTok{# R is case sensitive}
\NormalTok{Weight_kg <-}\StringTok{ }\DecValTok{60}
\NormalTok{Weight_kg <-}\StringTok{ }\DecValTok{30}


\CommentTok{#we can do arithmetic with R objects}
\FloatTok{2.2} \NormalTok{*}\StringTok{ }\NormalTok{weight_kg}

\CommentTok{#modify R object by reassigning object to new number}
\NormalTok{weight_kg <-}\StringTok{ }\FloatTok{57.5}
\FloatTok{2.2} \NormalTok{*}\StringTok{ }\NormalTok{weight_kg}

\CommentTok{#we can re-name 2.2*weight_kg as weight_lb}
\NormalTok{weight_lb <-}\StringTok{ }\FloatTok{2.2} \NormalTok{*}\StringTok{ }\NormalTok{weight_kg}

\CommentTok{#change value of weight_kg}
\NormalTok{weight_kg <-}\StringTok{ }\DecValTok{100}
\end{Highlighting}
\end{Shaded}

\subsubsection{Working with the
Environment}\label{working-with-the-environment}

You can see what objects (variables) are stored by viewing the
Environment tab in Rstudio. You can also use the \texttt{ls()} function.
You can remove objects (variables) with the \texttt{rm()} function. You
can do this one at a time or remove several objects at once. You can
also use the little broom button in your environment pane to remove
everything from your environment.

\begin{Shaded}
\begin{Highlighting}[]
\CommentTok{#list objects in environment}
\KeywordTok{ls}\NormalTok{()}

\CommentTok{#remove objects from environment}
\KeywordTok{rm}\NormalTok{(weight_lb, weight_kg)}
\KeywordTok{ls}\NormalTok{()}

\NormalTok{weight_lb <-}\StringTok{ }\DecValTok{20}
\NormalTok{weight_lb}

\CommentTok{#Removes all objects from the environment}
\KeywordTok{rm}\NormalTok{(}\DataTypeTok{list =} \KeywordTok{ls}\NormalTok{())}
\end{Highlighting}
\end{Shaded}

\subsubsection{EXERCISE 1}\label{exercise-1}

\begin{enumerate}
\def\labelenumi{\arabic{enumi}.}
\tightlist
\item
  You have a patient with a height (inches) of 73 and a weight (lbs) of
  203. Create r objects labeled `height' and `weight'.
\end{enumerate}

\begin{Shaded}
\begin{Highlighting}[]
\NormalTok{height <-}\StringTok{ }\DecValTok{73}
\NormalTok{weight <-}\StringTok{ }\DecValTok{203}
\end{Highlighting}
\end{Shaded}

\begin{enumerate}
\def\labelenumi{\arabic{enumi}.}
\setcounter{enumi}{1}
\tightlist
\item
  Convert `weight' to `weight\_kg' by dividing by 2.2. Convert `height'
  to `height\_m' by dividing by 39.37
\end{enumerate}

\begin{Shaded}
\begin{Highlighting}[]
\NormalTok{weight_kg <-}\StringTok{ }\NormalTok{weight /}\FloatTok{2.2}
\NormalTok{height_m <-}\StringTok{ }\NormalTok{height/}\FloatTok{39.37}
\end{Highlighting}
\end{Shaded}

\begin{enumerate}
\def\labelenumi{\arabic{enumi}.}
\setcounter{enumi}{2}
\tightlist
\item
  Calculate a new object `bmi' where BMI = weight\_kg /
  (height\_m*height\_m)
\end{enumerate}

\begin{Shaded}
\begin{Highlighting}[]
\NormalTok{bmi <-}\StringTok{ }\NormalTok{weight_kg /}\StringTok{ }\NormalTok{(height_m *}\StringTok{ }\NormalTok{height_m)}
\NormalTok{bmi <-}\StringTok{ }\NormalTok{weight_kg /}\StringTok{ }\NormalTok{(height_m ^}\StringTok{ }\DecValTok{2}\NormalTok{)}
\end{Highlighting}
\end{Shaded}

\section{--------------------------------------------------}\label{section}

\section{Built-in Functions}\label{built-in-functions}

R has built-in functions that allow you to do more than simple math.
Functions take in arguments within the parenthesis and return a value of
a specific type.

\begin{Shaded}
\begin{Highlighting}[]
\KeywordTok{sqrt}\NormalTok{(}\DecValTok{144}\NormalTok{)}
\KeywordTok{log}\NormalTok{(}\DecValTok{1000}\NormalTok{)}
\end{Highlighting}
\end{Shaded}

\section{Get help with function}\label{get-help-with-function}

To get help with a function (or anything in R) you can use the
\texttt{?} mark followed by the name of the function.

\begin{Shaded}
\begin{Highlighting}[]
\NormalTok{?log}
\end{Highlighting}
\end{Shaded}

Note syntax highlighting when typing this into the editor. Also note how
we pass \emph{arguments} to functions. The \texttt{base=} part inside
the parentheses is called an argument, and most functions use arguments.
Arguments modify the behavior of the function. Functions some input
(e.g., some data, an object) and other options to change what the
function will return, or how to treat the data provided. Finally, see
how you can \emph{next} one function inside of another (here taking the
square root of the log-base-10 of 1000).

\begin{Shaded}
\begin{Highlighting}[]
\KeywordTok{log10}\NormalTok{(}\DecValTok{1000}\NormalTok{)}
\KeywordTok{log}\NormalTok{(}\DecValTok{1000}\NormalTok{, }\DataTypeTok{base =} \DecValTok{10}\NormalTok{)}
\CommentTok{#The base= part inside the parentheses is called an argument. Arguments are the inputs to functions}
\end{Highlighting}
\end{Shaded}

We can write functions without labeling arguments as long as they come
in the same order as in the help file

\begin{Shaded}
\begin{Highlighting}[]
\KeywordTok{log}\NormalTok{(}\DecValTok{1000}\NormalTok{, }\DecValTok{10}\NormalTok{)}
\end{Highlighting}
\end{Shaded}

\section{Nesting Functions}\label{nesting-functions}

Oftentimes you will want to do several functions at once, which allows
you to use less lines of code. Nesting functions allows you to use the
output of one function as the input of the next function. This works as
long as the type of output returned from the first function is the same
as the type of input allowed from the second function. (e.g.~both are
numeric, both are characters, etc.)

\begin{Shaded}
\begin{Highlighting}[]
\KeywordTok{sqrt}\NormalTok{(}\KeywordTok{log}\NormalTok{(}\DecValTok{1000}\NormalTok{, }\DecValTok{10}\NormalTok{))}
\CommentTok{#Because sqrt() takes a number and because log() outputs a number we can nest the two together}

\CommentTok{#create intermediate object to make nesting easier to decipher}
\NormalTok{myval <-}\StringTok{ }\KeywordTok{log}\NormalTok{(}\DecValTok{1000}\NormalTok{, }\DecValTok{10}\NormalTok{)}
\KeywordTok{sqrt}\NormalTok{(myval)}
\end{Highlighting}
\end{Shaded}

\section{------------------------------------------------}\label{section-1}

\subsubsection{EXERCISE 2}\label{exercise-2}

See \texttt{?abs} and calculate the square root of the log-base-10 of
the absolute value of \texttt{-4*(2550-50)}. Answer should be
\texttt{2}.

\begin{Shaded}
\begin{Highlighting}[]
\KeywordTok{sqrt}\NormalTok{(}\KeywordTok{log}\NormalTok{(}\KeywordTok{abs}\NormalTok{(-}\DecValTok{4}\NormalTok{*(}\DecValTok{2550} \NormalTok{-}\StringTok{ }\DecValTok{50}\NormalTok{)), }\DecValTok{10}\NormalTok{))}
\end{Highlighting}
\end{Shaded}

\section{------------------------------------------------}\label{section-2}

\subsection{Vectors}\label{vectors}

Vectors are one of the most basic data types in R. They are a sequence
of data elemnets of the same type. This can be numeric (2, 3.2, etc.),
character (``a'', ``a sentence is here'', ``548-4241''), logical
(``TRUE'', ``FALSE''), etc, they just all have to be of the same type.

Using the \texttt{:} operator allows you to create a consecutive
sequence of numbers.

\begin{Shaded}
\begin{Highlighting}[]
\DecValTok{1}\NormalTok{:}\DecValTok{5}
\DecValTok{6}\NormalTok{:}\DecValTok{10}
\DecValTok{1}\NormalTok{:}\DecValTok{5} \NormalTok{+}\StringTok{ }\DecValTok{6}\NormalTok{:}\DecValTok{10}

\CommentTok{#Careful, they must be multiples of each other}
\CommentTok{#1:5 + 6:8 #(Will not work)}
\DecValTok{1}\NormalTok{:}\DecValTok{6} \NormalTok{+}\StringTok{ }\DecValTok{6}\NormalTok{:}\DecValTok{8}

\CommentTok{#This will add 3 to each element of the vector}
\DecValTok{1}\NormalTok{:}\DecValTok{5} \NormalTok{+}\StringTok{ }\DecValTok{3}
\end{Highlighting}
\end{Shaded}

To create vectors that not necessilary consecutive you use the c()
function (concatenate / combine). This allows you to enter in as many
different values into a vector.

\begin{Shaded}
\begin{Highlighting}[]
\KeywordTok{c}\NormalTok{(}\DecValTok{1}\NormalTok{, }\DecValTok{5}\NormalTok{, }\DecValTok{6}\NormalTok{)}
\KeywordTok{c}\NormalTok{(}\DecValTok{1}\NormalTok{:}\DecValTok{5}\NormalTok{, }\DecValTok{1}\NormalTok{:}\DecValTok{6}\NormalTok{)}

\CommentTok{#What if we wanted to created a vector from to 2 to 200 by 4s?}
\end{Highlighting}
\end{Shaded}

The seq function allows you to create a vector of values that are
increase from one value to another by a designated increment.

\begin{Shaded}
\begin{Highlighting}[]
\NormalTok{?seq}

\KeywordTok{seq}\NormalTok{(}\DataTypeTok{from =} \DecValTok{2}\NormalTok{, }\DataTypeTok{to =} \DecValTok{200}\NormalTok{, }\DataTypeTok{by =} \DecValTok{4}\NormalTok{)}
\end{Highlighting}
\end{Shaded}

Just like other objects, you can assign a vector a name.

\begin{Shaded}
\begin{Highlighting}[]
\CommentTok{#assign vectors to object name}
\NormalTok{animal_weights <-}\StringTok{ }\KeywordTok{c}\NormalTok{(}\DecValTok{50}\NormalTok{, }\DecValTok{60}\NormalTok{, }\DecValTok{66}\NormalTok{)}
\NormalTok{animal_weights}

\CommentTok{#vectors can also be character type}
\NormalTok{animals <-}\StringTok{ }\KeywordTok{c}\NormalTok{(}\StringTok{"mouse"}\NormalTok{, }\StringTok{"rat"}\NormalTok{, }\StringTok{"dog"}\NormalTok{)}
\NormalTok{animals}
\end{Highlighting}
\end{Shaded}

When creating vectors or looking at data from other sources, you may
want to inspect the vector more closely. You will want to know things
like how long vector is, the number elements, which type of object it
is, etc. There are a series of functions that can be used to inspect
vectors.

\begin{Shaded}
\begin{Highlighting}[]
\CommentTok{#Inspecting vectors}
\CommentTok{#Length provides the number of elements}
\KeywordTok{length}\NormalTok{(animals)}

\CommentTok{#Class shows the type of vector}
\KeywordTok{class}\NormalTok{(animals)}
\KeywordTok{class}\NormalTok{(animal_weights)}

\CommentTok{#str shows the structure of the vector}
\KeywordTok{str}\NormalTok{(animals)}
\KeywordTok{str}\NormalTok{(animal_weights)}
\end{Highlighting}
\end{Shaded}

Oftentimes we will want to add to a vector. To do this, we can again use
the c() function.

\begin{Shaded}
\begin{Highlighting}[]
\CommentTok{#appending to vectors using c()}
\NormalTok{animal_weights <-}\StringTok{ }\KeywordTok{c}\NormalTok{(animal_weights, }\DecValTok{80}\NormalTok{)}
\NormalTok{animal_weights <-}\StringTok{ }\KeywordTok{c}\NormalTok{(}\DecValTok{49}\NormalTok{, animal_weights)}
\NormalTok{animal_weights}
\end{Highlighting}
\end{Shaded}

We can also use functions on vectors, just like we did before.

\begin{Shaded}
\begin{Highlighting}[]
\CommentTok{#Make sure that you are using the correct input type}
\CommentTok{#sum(animals)}
\KeywordTok{sum}\NormalTok{(animal_weights)}
\KeywordTok{mean}\NormalTok{(animal_weights)}
\end{Highlighting}
\end{Shaded}

One key thing to learn in R is how to access individual elements within
a vector. Sometimes you might just want to look at the first 5 elements
of a vector or you want to return the 8th, 12th, and 20th elements. To
access these elements, we will use brackets {[}{]} and the c() function.

\begin{Shaded}
\begin{Highlighting}[]
\CommentTok{#Indexing vectors}
\NormalTok{x <-}\StringTok{ }\DecValTok{100}\NormalTok{:}\DecValTok{150}
\CommentTok{#Using [1] allows you to see the very first element of a vector}
\NormalTok{x[}\DecValTok{1}\NormalTok{]}
\NormalTok{animals[}\DecValTok{1}\NormalTok{]}

\CommentTok{#You can also look through a sequence of elements.}
\CommentTok{#fifth through 10th elements}
\NormalTok{x[}\DecValTok{5}\NormalTok{:}\DecValTok{10}\NormalTok{]}
\NormalTok{animals[}\DecValTok{2}\NormalTok{:}\DecValTok{3}\NormalTok{]}

\CommentTok{#Using the c() function allows you to look at the elements in a non-sequential manner.}
\CommentTok{#40th and 48th elements (non-sequential)}
\NormalTok{x[}\KeywordTok{c}\NormalTok{(}\DecValTok{40}\NormalTok{, }\DecValTok{48}\NormalTok{)]}
\NormalTok{animals[}\KeywordTok{c}\NormalTok{(}\DecValTok{1}\NormalTok{, }\DecValTok{3}\NormalTok{)] }

\CommentTok{#what happens when we call an index beyond the vector}
\NormalTok{x[}\DecValTok{1}\NormalTok{:}\DecValTok{150}\NormalTok{]}
\end{Highlighting}
\end{Shaded}

You can also use the indexing features to add or change elements within
a vector by placing the vector on the left hand side of the assignment
operator.

Adding an element to a vector will increase its size by 1 and will
increment the position (if not placed at the end) by 1 of any element
that follows.

\begin{Shaded}
\begin{Highlighting}[]
\NormalTok{animals[}\KeywordTok{c}\NormalTok{(}\DecValTok{4}\NormalTok{)] <-}\StringTok{ "lizard"}

\NormalTok{animals[}\KeywordTok{c}\NormalTok{(}\DecValTok{15}\NormalTok{)] <-}\StringTok{ "bat"}
\NormalTok{animals}
\NormalTok{animals[}\KeywordTok{c}\NormalTok{(}\DecValTok{1500}\NormalTok{)] <-}\StringTok{ "horse"}
\NormalTok{animals}
\NormalTok{animals[}\KeywordTok{c}\NormalTok{(}\DecValTok{1}\NormalTok{, }\DecValTok{3}\NormalTok{)] <-}\StringTok{ }\KeywordTok{c}\NormalTok{(}\StringTok{"anteater"}\NormalTok{, }\StringTok{"koala"}\NormalTok{)}
\end{Highlighting}
\end{Shaded}

\section{--------------------------------------------------}\label{section-3}

\subsection{Data frames}\label{data-frames}

Data frames store heterogeneous tabular data in R: tabular, meaning that
individuals or observations are typically represented in rows, while
variables or features are represented as columns; heterogeneous, meaning
that columns/features/variables can be different classes (on variable,
e.g.~age, can be numeric, while another, e.g., cause of death, can be
text)

The first step is reading in data from an external source. The first way
we will do that is through the read.csv function which allows you to
read in a csv file as a data frame. The first argument is the file name
and the second option \texttt{header\ =} allows you to specify whether
column names are present in the first column of the csv file or not.

\begin{Shaded}
\begin{Highlighting}[]
\NormalTok{##load data using read.csv}
 \NormalTok{gm <-}\StringTok{ }\KeywordTok{read.csv}\NormalTok{(}\StringTok{"Data/gapminder.csv"}\NormalTok{, }\DataTypeTok{header =} \OtherTok{TRUE}\NormalTok{)}
 
\CommentTok{#another option is read_csv in the readr package}
\end{Highlighting}
\end{Shaded}

As with a vector, one of the most important things to do is to inspect
your dataframe to ensure that everything was read in in a way that makes
sense.

\begin{Shaded}
\begin{Highlighting}[]
\CommentTok{#Class shows the type of object}
\KeywordTok{class}\NormalTok{(gm)}

\CommentTok{#Head shows the first group of rows and tail the last bunch of rows (the number can be changed in the options)}
\KeywordTok{head}\NormalTok{(gm)}
\KeywordTok{tail}\NormalTok{(gm, }\DecValTok{10}\NormalTok{)}

\CommentTok{#Dim provides the number of rows and columns, nrow the number of rows, and ncol the number of columns.}
\KeywordTok{dim}\NormalTok{(gm)}
\KeywordTok{nrow}\NormalTok{(gm)}
\KeywordTok{ncol}\NormalTok{(gm)}

\CommentTok{#The names function shows the column/variable names}
\KeywordTok{names}\NormalTok{(gm)}

\CommentTok{#The summary function provides a quick descriptive overview of each column. The output is based upon the type of data that is stored in the column.}
\KeywordTok{summary}\NormalTok{(gm)}

\CommentTok{#The str function shows the detailed structure of the dataframe}
\KeywordTok{str}\NormalTok{(gm)}
\end{Highlighting}
\end{Shaded}

To access a single column in a dataframe, you use the dataset name
followed by the \texttt{\$} symbol and then the name of the dataframe
you want to look at.

\begin{Shaded}
\begin{Highlighting}[]
\CommentTok{#Using the $ to access variables}
\NormalTok{gm$pop}

\CommentTok{#This allows you to provide a function that simply looks at one column (just like we did with a vector)}
\CommentTok{#The na.rm function removes any missing values from the analysis, since the result of this function would be na if any missing values were in that variable.}
\KeywordTok{mean}\NormalTok{(gm$lifeExp, }\DataTypeTok{na.rm =} \OtherTok{TRUE}\NormalTok{)}
\end{Highlighting}
\end{Shaded}

\section{------------------------------------------------------}\label{section-4}

\subsubsection{EXERCISE 3}\label{exercise-3}

\begin{enumerate}
\def\labelenumi{\arabic{enumi}.}
\tightlist
\item
  What's the standard deviation of the life expectancy (hint: get help
  on the \texttt{sd} function with \texttt{?sd}).
\end{enumerate}

\begin{Shaded}
\begin{Highlighting}[]
 \NormalTok{?sd}
 \KeywordTok{sd}\NormalTok{(gm$lifeExp, }\DataTypeTok{na.rm =} \OtherTok{TRUE}\NormalTok{)}
\end{Highlighting}
\end{Shaded}

\begin{enumerate}
\def\labelenumi{\arabic{enumi}.}
\setcounter{enumi}{1}
\tightlist
\item
  What's the mean population size in millions? (hint: divide by
  \texttt{1000000}, or alternatively, \texttt{1e6}).
\end{enumerate}

\begin{Shaded}
\begin{Highlighting}[]
 \KeywordTok{mean}\NormalTok{(gm$pop, }\DataTypeTok{na.rm =} \OtherTok{TRUE}\NormalTok{) /}\StringTok{ }\FloatTok{1e6}
\end{Highlighting}
\end{Shaded}

\begin{enumerate}
\def\labelenumi{\arabic{enumi}.}
\setcounter{enumi}{2}
\tightlist
\item
  What's the range of years represented in the data? (hint:
  \texttt{range()}).
\end{enumerate}

\begin{Shaded}
\begin{Highlighting}[]
 \KeywordTok{range}\NormalTok{(gm$year)}
 \KeywordTok{range}\NormalTok{(gm$pop)}
\end{Highlighting}
\end{Shaded}

\begin{enumerate}
\def\labelenumi{\arabic{enumi}.}
\setcounter{enumi}{3}
\tightlist
\item
  Run a correlation between life expectancy and GDP per capita (hint:
  ?cor())
\end{enumerate}

\begin{Shaded}
\begin{Highlighting}[]
 \KeywordTok{cor}\NormalTok{(gm$lifeExp, gm$gdpPercap, }\DataTypeTok{method =} \StringTok{"kendall"}\NormalTok{)}
\end{Highlighting}
\end{Shaded}

\href{Week_1_RMarkdown.html}{Let's move on to learning about using R
Markdown}.


\end{document}
